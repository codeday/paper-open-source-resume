\section{Background}

\subsection{Employers Describe a Need for Skills Acquired in Work-Based Learning}

Skill surveys of industry representatives inform the alignment of class content with relevant industry skills. Such surveys are regularly conducted in CS \citep[e.g.,][]{yahyaMappingGraduateSkills2024, stepanovaHiringCSGraduates2021, scaffidiEmployersNeedsComputer2018, scaffidiSurveyEmployersNeeds2018}. Although many surveys focus exclusively on specific technical skills that can be taught in the classroom, several studies have found that, for CS and other STEM students, employers place a high value on cross-cutting technical skills, the ability to work independently, and ``soft skills'' \citep{scaffidiSurveyEmployersNeeds2018,raynerEmployerPerspectivesCurrent2015,cheangEmployersExpectationsUniversity2023,menezesWhatSkillsCS2023,humeAreWeDeveloping2024,stalhaneWhatCompetenceSoftware2020}. This trend is also evident to students, who report a desire for more career-related training \citep{gedyeStudentsUndergraduateExpectations2004,craigListeningEarlyCareer2018}, and to recent graduates, who report a disconnect between what they learned in college and their first job \citep{begelStrugglesNewCollege2008,kapoorUnderstandingCSUndergraduate2019}.

The skills students learn in the classroom can, as \citet{beaubouefComputerScienceCurriculum2011} put it, ``simply enable [students] to be able to learn what they need to know later on.'' Industry experiences help students develop soft skills and learn to research, experiment, and make decisions independently. For this reason, many undergraduate CS degrees now integrate industry engagement opportunities such as guest speakers, industry mentoring, and case studies. Some programs have even experimented with industry partnerships to provide credit for ``on-the-job'' learning equivalents to classes like data structures, web apps, and databases \citep{carmichaelCurriculumAlignedWorkIntegratedLearning2018}. Internships, however, are the most common form of work-based learning in CS.

It should be no surprise, then, that internships can significantly influence how easily recent graduates transition into the workforce. In fact, whether a student has undertaken an internship is one of the most important variables predicting whether, and how quickly, they have a job after graduation \citep{callananAssessingRoleInternships2004, saltikoffPositiveImplicationsInternships2017, knouseRelationCollegeInternships1999}.


\subsection{Barriers to Internship Access and Open Source as an Alternative}

Unfortunately, internships are scarce. A recent report from the Business--Higher Education Forum found that the supply of internships is insufficient \citep{bhefExpandingInternshipsHarnessing2024}. Of CS undergraduates, only 20\% of rising sophomores and under half of rising juniors/seniors secure internships. By graduation, just 60\% have participated in an internship \citep{kapoorExploringParticipationCS2020, kocClass2014Student2014}. They are also less accessible to lower-income students: most students with a household income over \$150,000 per year graduate with an internship, versus only 35\% of students with a household income under \$100,000 per year \citep{kapoorExploringParticipationCS2020}. There are several reasons why access to internships is difficult for lower-income students:
\begin{itemize}
    \item They may not be able to commit to an internship because they are a primary caregiver or must work a part-time job to pay for school \citep{kapoorBarriersSecuringIndustry2020}.

    \item They may not apply due to lack of confidence \citep{kapoorBarriersSecuringIndustry2020}.

    \item They may not have time to take necessary actions to prepare and apply: successful students spend 3 hours weekly on applications versus 1 hour for unsuccessful ones \citep{kapoorBarriersSecuringIndustry2020,kapoorExploringParticipationCS2020}.

    \item They may attend smaller, more affordable schools that have a harder time capturing the attention of recruiters.
\end{itemize}
Traditional internships are also difficult to integrate into the classroom, and they occur late in students' educational pathways \citep{luceroStructureCharacteristicsSuccessful2021,martincicCombiningRealWorldInternships2009}.

In response to these deficiencies, many colleges have experimented with having students contribute to open source software projects \citep{choiOpenSourceSoftware2021,mullerEngagingStudentsOpen2019,hangIndustryMentoringInternship2024,menezesOpenSourceInternshipsIndustry2022,lovellContributorCatalystPilot2024,lovellScaffoldingStudentSuccess2021,dossantosmontagnerLearningProfessionalSoftware2022,salernoBarriersSelfEfficacyLargeScale2023,hislopStudentReflectionsLearning2020}. Open source software is licensed so that users can read, change, and redistribute its source code\citep{OpenSourceDefinition}. Moreover, anyone is welcome to contribute changes to the original project, where they may be used by others. Popular open source projects include Linux (the operating system kernel), React (the JavaScript library for building user interfaces developed by Meta), and WordPress (a content management system that powers over 40\% of websites). These projects have communities of thousands to hundreds of thousands of contributors, some paid and some who donate their time.

Educators consider many factors when bringing open source into the classroom \citep{nascimentoUsingOpenSource2013}, but few studies have focused on how employers view these open source contributions.

\subsection{Frameworks for Student Motivation}

We considered several frameworks to undergird our investigation into student motivation. All emphasize the role of beliefs and expectations in motivation and career development, though they approach these concepts from different angles.

One of the largest barriers to participation in work-based learning opportunities is low self-efficacy. \citet{hackettSelfefficacyApproachCareer1981} developed a model of career decision self-efficacy that identified four main factors: (1) performance accomplishments, (2) vicarious learning, (3) emotional state, and (4) verbal persuasion.

Social cognitive career theory (SCCT) suggests that individuals are more motivated to undertake activities that might advance their careers and to overcome setbacks when they believe they are competent and have a chance at a positive outcome \citep{lentUnifyingSocialCognitive1994}. SCCT provides a more comprehensive explanation of career development, incorporating social and environmental factors alongside individual beliefs. It also emphasizes the reciprocity among these influences---how experiences shape beliefs, which influence choices and, in turn, create new experiences.

Expectancy‑-value theory (EVT) \citep{wigfieldExpectancyvalueTheory2009} posits that students' achie\-vement related choices are influenced by multiple interconnected factors including social and cultural contexts, affective responses, and individual beliefs about the value of academic tasks. Specifically, EVT examines how students' perceptions of achievement value---encompassing intrinsic value, attainment value, and utility value---along with their assessment of participation costs, collectively shape their engagement in particular academic tasks or activities \citep{ecclesMotivationalBeliefsValues2002,wigfieldDevelopmentCompetenceBeliefs2002,wigfieldExpectancyvalueTheory2009}. EVT focuses on the motivational components of specific tasks or domains, making it particularly useful for understanding moment-to-moment engagement and effort.

We chose EVT because of its wide acceptance in undergraduate educational contexts, and because research suggests that both expectancy and value predict career-related decisions \citep{renningerCambridgeHandbookMotivation2019}.