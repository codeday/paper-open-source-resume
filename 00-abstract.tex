Computer science educators are increasingly integrating open sou\-rce contributions into classes to prepare students for higher expectations due to GenAI, and to improve employment outcomes in an increasingly competitive job market. However, little is known about how employers view student open source contributions.

This paper addresses two research questions qualitatively: what traits do employers desire for entry-level hires in 2025, and how can they be demonstrated through open source contributions? It also tests quantitatively the hypothesis that student knowledge of employers' expectations will improve their motivation to work on open source projects.

To answer our qualitative questions, we conducted interviews with US hiring managers. We collaborated with each interviewee to create a ``hiring manager agreement,'' which listed desirable traits and specific ways to demonstrate them through open source, along with a promise to interview some students meeting the criteria. To evaluate our quantitative hypothesis, we surveyed 650 undergraduates attending public universities in the US using an instrument based on expectancy--value theory.

Hiring managers wanted many non-technical traits that are difficult to teach in traditional CS classes, such as initiative. There were many commonalities in how employers wanted to see these traits demonstrated in open source contributions. Viewing hiring manager agreements improved student motivation to contribute to open source projects.

Our findings suggest that open source contributions may help CS undergraduates get hired, but this requires sustained engagement in multiple areas. Educators can motivate students by sharing employer expectations, but further work is required to determine if this changes their behavior.