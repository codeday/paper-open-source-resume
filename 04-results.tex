\section{Results}
\subsection{Employer Interviews}

\pgfplotstableread[row sep=\\,col sep=&]{
    theme & prevalence \\
    Persistence & 8 \\
    Product Sense & 10 \\
    Curiosity & 9 \\
    Domain Knowl. & 10 \\
    Teamwork & 13 \\
    Initiative & 14 \\
    Problem Solving & 17 \\
}\themesData

\begin{figure}
    \Description[Bar chart showing theme prevalence in hiring manager agreements]{Bar chart showing theme prevalence in hiring manager agreements. Problem solving and initiative were the most common themes, while teamwork and domain knowledge were seen in about half of agreements. Curiosity and product sense were also included in a small number of agreements.}
    \centering
    \hspace{0.6cm}
    \begin{tikzpicture}
        \begin{axis}[
                xbar,
                symbolic y coords={Persistence,Curiosity,Product Sense,Domain Knowl.,Teamwork,Initiative,Problem Solving},
                ytick=data,
                xmin=0,
                xmax=20,
                xlabel={Number of responses (out of 20)},
                y=0.6cm,
                width=7cm,
                axis lines = left,
                enlarge y limits={0.1},
                nodes near coords={\pgfmathprintnumber\pgfplotspointmeta},
                xticklabel={$\pgfmathprintnumber{\tick}$},
                ]
            \addplot [fill=black] table[x=prevalence,y=theme]{\themesData};
        \end{axis}
    \end{tikzpicture}
    \caption{Prevalence of themes in hiring manager agreements}
    \label{fig:themesChart}
\end{figure}
\begin{table*}
\caption{Themes in hiring manager agreements and corresponding examples of open source demonstration}
\label{tab:themes}
\begin{tabular}{p{2cm}p{15cm}}
\toprule
\textbf{Theme} & \textbf{Open Source Evidence Examples} \\ \midrule
\begin{tabular}[c]{@{}l@{}}Problem\\ Solving\end{tabular} & \begin{tabular}[c]{@{}p{14cm}}``10 issues solved; a sufficient variety of types of issues they tackle; issues touch on different parts of the technology stack or solve different kinds of problems.; at least one issue should be labeled as medium or hard.''\vspace{0.15cm} \\ ``6 issues solved. The issues should be reasonably challenging (required a change across multiple modules)''\end{tabular} \\ \midrule
Initiative & \begin{tabular}[c]{@{}p{14cm}}``1 example of solving a problem for the project that wasn’t directly asked for (e.g., they open and solve an issue themselves, initiate a new feature, or improve the project’s technical documentation); 2 examples of helping onboard someone into the project (e.g., answering questions on the project’s forum).''\vspace{0.15cm} \\ ``2 examples of a PR where the definition of `done' goes beyond the bare minimum of what is asked for in the original issue. Do they think about details that are not explicitly asked for (e.g., writing unit tests or updating documentation)? Do they consider other aspects of the product, like performance or security?''\end{tabular} \\ \midrule
\begin{tabular}[c]{@{}l@{}}Teamwork/\\ Collaboration\end{tabular} & \begin{tabular}[c]{@{}p{14cm}}``2 examples of comments on a PR that is not their own, where their feedback is accepted by the other person. 2 examples of a pull request where the candidate received critical feedback and they incorporated the feedback.''\vspace{0.15cm} \\ ``2 examples of thorough documentation on a pull request: document the process of how they initiated the new issue/feature. [...] Even better if the issue has comments throughout the timeline of the Github thread.\end{tabular} \\ \midrule
\begin{tabular}[c]{@{}l@{}}Domain\\ Knowledge\end{tabular} & \begin{tabular}[c]{@{}p{14cm}}``Experience with tools that SRE’s might use: Data Dog, Grafana, Splunk, Prometheus''\vspace{0.15cm} \\ ``Jupyter notebooks (API’s), Docker/Kubernetes, Node, Unix Shell''\end{tabular} \\ \midrule
Curiosity & \begin{tabular}[c]{@{}p{14cm}}``2 examples of finding and filing new bugs on the project. 1 example of fixing a complex bug (getting to the root cause), where they had to explore multiple branches of possibilities. The solution is explained in a blog post, video walkthrough, or detailed pull request description.''\vspace{0.15cm} \\ ``2 examples of a PR where they consider multiple approaches to solving the problem. Describe the multiple possibilities in the pull request. Bonus points for making a case for your best recommendation.''\end{tabular} \\ \midrule
\begin{tabular}[c]{@{}l@{}}Product\\ Sense\end{tabular} & \begin{tabular}[c]{@{}p{14cm}}``You can state your opinion on what new features should be added to an open source project you are involved in and explain why your recommendation would solve a problem for the customers/users of the product''\vspace{0.15cm} \\ ``You can clearly articulate the purpose of the project. You can name specific projects/organizations who use it and/or you can describe the primary use cases for a typical end user.'\end{tabular} \\ \midrule
Persistence & \begin{tabular}[c]{@{}p{14cm}}``Time to completion averages to 3 months or less''\vspace{0.15cm} \\ ``1 example of a long-running PR where you follow through to the end. Even better if the issue has comments throughout the timeline of the PR.''\end{tabular} \\ \bottomrule
\end{tabular}
\end{table*}

A total of 20 interviews were conducted with hiring managers at technology companies. While theoretical saturation was achieved after 12 interviews, 8 additional interviews were completed to strengthen the robustness of the findings and address reviewer recommendations. Among the interviewees, 5 were C-level executives, 2 were senior leadership, and the remaining 13 were mid-to-senior-level managers.

Hiring managers were willing to interview various numbers of students. The smallest commitment was 3 interviews (2 hiring managers) and the largest was 10 (1 hiring manager); all the remaining hiring managers (17) were willing to interview 5 students.

When asked about the traits they desired, 7 common themes were identified (Figure~\ref{fig:themesChart}, Table~\ref{tab:themes}):  \textbf{problem solving} (ability to solve complex problems without hand-holding), \textbf{initiative} (solving problems without being asked), \textbf{teamwork/collaboration} (communication and coachability), \textbf{domain knowledge} (specific technologies used at the company), \textbf{curiosity} (learning for the sake of it), \textbf{product sense} (understanding the product and users), and \textbf{persistence} (working until a problem is solved).

\subsection{Student Motivation}

We recruited 650 students to complete the motivation survey; 591 (the control group) did not view the hiring manager agreements, and 59 (the experimental group) did. Students were recruited from juniors and seniors at state-funded universities and community and technical colleges in the US.

Data normality was assessed using the Shapiro--Wilk test for each construct within both groups. The normality assumption was satisfied for all constructs ($p > 0.05$). Data for some constructs were found to violate equal-variance assumptions, so Welch $t$ tests were performed to compare means across groups for each construct. To control for multiple comparisons, $p$ values were adjusted using the Holm--Bonferroni method.

The results (Figures \ref{fig:value} and~\ref{fig:cost} and Table~\ref{tab:stats}) revealed statistically significant differences between students who had and had not viewed the hiring manager agreements across all constructs ($p < 0.01$ to $p < 0.0001$) except Opportunity and Emotional Costs. Effect size analyses indicated a small practical effect for Utility Value; a moderate effect for Intrinsic Value, Effort Cost, and Outside Effort Cost; and a large effect for Attainment Value ($d = 2.17$). All significant differences favored increased motivation among students who viewed the agreements.

\pgfplotstableread[row sep=\\,col sep=&]{
    construct   & baseline  & agreement & baselineErr   & agreementErr\\
    intrinsic   & 3.87      & 4.21	    & 0.06          & 0.24 \\
    attainment  & 3.46      & 4.41      & 0.05          & 0.18 \\
    utility     & 3.78      & 4.10      & 0.13          & 0.23 \\
}\valueData

\begin{figure}
    \centering
    \hspace{0.6cm}
    \begin{tikzpicture}
        \begin{axis}[
                xbar,
                symbolic y coords={utility,attainment,intrinsic},
                ytick=data,
                xmin=1,
                xmax=5,
                y=1.18cm,
                width=7cm,
                axis lines = left,
                enlarge y limits={0.34},
                nodes near coords={\pgfmathprintnumber\pgfplotspointmeta},
                nodes near coords align={shift={(0.55cm,0)}},
                xtick={1,2,3,4,5},
                xticklabel style={align=center},
                xmajorgrids=true,
                reverse legend,
                legend style={at={(0.5,1.3)},
                    anchor=north,legend columns=-1
                    ,inner sep=3pt,column sep=3pt},
                ]
            \addplot [fill=gray!30]
            plot [nodes near coords align={shift={(0.63cm,0)}}, error bars/.cd, x dir=both, x explicit]
            table[x=agreement,y=construct,x error=agreementErr] {\valueData};
            \addplot [black]
            plot [nodes near coords align={shift={(0.5cm,0)}}, error bars/.cd, x dir=both, x explicit]
            table[x=baseline,y=construct,x error=baselineErr]{\valueData};
            \legend{viewed agreements,did not view agreements}
        \end{axis}
    \end{tikzpicture}
    \caption{Comparison of motivational beliefs about value constructs by hiring manager agreement viewing status: higher indicates more motivated (99\% C.I., $n$ = 650)}    \Description[Bar chart showing motivational beliefs about value constructs]{Bar chart showing motivational beliefs about value constructs by agreement viewing status, with 99\% confidence intervals. Intrinsic and attainment beliefs increased beyond confidence interval bounds after viewing agreements, while utility increased within the bounds.}
    \label{fig:value}
\end{figure}
\pgfplotstableread[row sep=\\,col sep=&]{
    construct       & baseline  & agreement & baselineErr   & agreementErr \\
    effort          & 3.28      & 2.75      & 0.11          & 0.29 \\
    outside effort  & 3.07      & 2.52      & 0.10          & 0.30  \\
    opportunity     & 2.46      & 2.33      & 0.07          & 0.28 \\
    emotional       & 2.11      & 2.28      & 0.08          & 0.28 \\
}\costData

\begin{figure}
    \centering
    \hspace{0.6cm}
    \begin{tikzpicture}
        \begin{axis}[
                xbar,
                symbolic y coords={emotional,opportunity,outside effort,effort},
                ytick=data,
                xmin=1,
                xmax=5,
                y=1.1cm,
                width=7cm,
                axis lines = left,
                enlarge y limits={0.15},
                nodes near coords={\pgfmathprintnumber\pgfplotspointmeta},
                nodes near coords,
                xtick={1,2,3,4,5},
                xticklabel style={align=center},
                xmajorgrids=true,
                reverse legend,
                legend style={at={(0.5,1.2)},
                    anchor=north,legend columns=-1
                    ,inner sep=3pt,column sep=3pt},
                ]
            \addplot [fill=gray!30]
            plot [nodes near coords align={shift={(0.8cm,0)}}, error bars/.cd, x dir=both, x explicit]
            table[x=agreement,y=construct,x error=agreementErr] {\costData};
            \addplot [black]
            plot [nodes near coords align={shift={(0.5cm,0)}}, error bars/.cd, x dir=both, x explicit]
            table[x=baseline,y=construct,x error=baselineErr]{\costData};
            \legend{viewed agreements,did not view agreements}
        \end{axis}
    \end{tikzpicture}
    \caption{Comparison of motivational beliefs about cost constructs by hiring manager agreement viewing status: lower indicates more motivated (99\% C.I., $n$ = 650)}
    \label{fig:cost}
    \Description[Bar chart showing motivational beliefs about cost constructs]{Bar chart showing motivational beliefs about cost constructs by agreement viewing status, with 99\% confidence intervals. Effort and outside effort beliefs decreased beyond confidence interval bounds after viewing agreements, while opportunity cost decreased within bounds, and emotional cost increased within bounds.}
\end{figure}
\begin{table}
\caption{Comparison of motivational beliefs by hiring manager agreement viewing status}

\label{tab:stats}
\begin{tabular}{lcccc}
\toprule
\textbf{} & \textbf{\begin{tabular}[c]{@{}c@{}}Not Viewed\\ Mean (SD)\end{tabular}} & \textbf{\begin{tabular}[c]{@{}c@{}}Viewed\\ Mean (SD)\end{tabular}} & \textbf{\begin{tabular}[c]{@{}l@{}}$t$\ test\end{tabular}} & \textbf{\begin{tabular}[c]{@{}c@{}}\small{Cohen's}\\ $d$\end{tabular}} \\
\midrule
\multicolumn{5}{c}{\textit{\textbf{Value (higher indicates more motivated)}}} \\
Intrinsic       & \(3.87\) (0.55)   & \(4.21\) (0.70)   & *    & \(0.61\)\\
Attainment      & \(3.46\) (0.43)   & \(4.40\) (0.51)   & ***  & \(2.17\)\\
Utility         & \(3.78\) (1.22)   & \(4.10\) (0.67)   & *    & \(0.27\)\\
\multicolumn{5}{c}{\textit{\textbf{Cost (lower indicates more motivated)}}} \\
Effort          & \(3.28\) (1.05)   & \(2.75\) (0.84)   & **    & \(-0.52\)\\
Outside Effort  & \(3.06\) (0.90)   & \(2.52\) (0.88)   & **    & \(-0.61\)\\
Opportunity     & \(2.46\) (0.68)   & \(2.33\) (0.81)   & n.s.  & n.s. \\
Emotional       & \(2.12\) (0.74)   & \(2.28\) (0.80)   & n.s.  & n.s.\\
\bottomrule
\end{tabular}

{\footnotesize * $p < 0.01$, **  $p<0.001$, ***  $p<0.0001$, n.s. = not statistically significant}
\end{table}