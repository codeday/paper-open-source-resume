\section{Introduction}

Despite the intrinsic value of education, most students pursuing an undergraduate degree are doing so to improve their job prospects \citep{gedyeStudentsUndergraduateExpectations2004,nortonPerceivedBenefitsUndergraduate2017}, and computer science (CS) students are no exception \citep{helpsStudentExpectationsComputing2005,alshahraniUsingSocialCognitive2018}. One of the most important factors predicting whether a student will obtain a job after graduation is whether they engaged in work-based learning, most commonly internships. However, many students do not participate in such activities because of competing life priorities or because their college is not a ``target'' school for university recruiters. In addition, the job market is becoming increasingly competitive for new graduates. AI coding tools now allow senior developers to quickly accomplish much of the work traditionally given to entry-level developers, so the expectations on new CS graduates are higher than ever.

Many educators are considering open source software contributions as a way to bring more work-based learning into the classroom \citep[e.g.,][]{nascimentoUsingOpenSource2013,hangIndustryMentoringInternship2024,marmorsteinOpenSourceContribution2011,mullerEngagingStudentsOpen2019,choiOpenSourceSoftware2021}. The authors run one such program, in which thousands of students have made their first contributions to open source software in a structured educational experience\identifying{ \citep{parraClosingGapClassrooms2021,narayananScalableApproachSupport2023,menezesOpenSourceInternshipsIndustry2022}}.

The goal of this work is to improve employment outcomes by grounding open source contribution objectives in employer expectations and improving student motivation. Accordingly, this paper addresses two qualitative research questions: \textbf{(RQ1)} What traits do employers desire for entry-level hires? \textbf{(RQ2)} How can students demonstrate those traits through open source work?
Also tested is the following quantitative hypothesis: \textbf{(H1)} Student knowledge of employers' desired traits and demonstrations through open source will improve their motivation to work on open source.