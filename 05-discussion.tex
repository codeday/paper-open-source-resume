\section{Discussion}

\subsection{RQ1: Employers' Desired Traits}

Of all the traits identified as important by hiring managers, only \textit{domain knowledge} (identified in 10 interviews) was easy to teach in lecture-focused courses. This demonstrates the importance of providing work-based learning opportunities.

The emphasis on \textit{initiative} (14 interviews), \textit{product sense} (10 interviews), and \textit{curiosity} (9 interviews) reflects a fundamental tension in CS education. Traditional coursework is largely assignment driven: students complete prescribed tasks with clear requirements and deadlines. However, employers want people who identify problems and solve them proactively and who are intrinsically motivated to grow their skills over time. This suggests that educators should create more open-ended learning experiences in which students must define their own problems to solve, rather than just implementing solutions to given specifications.

Two highly rated traits---\textit{problem solving} (17 interviews) and \textit{teamwork/collaboration} (13 interviews)---are, in the authors' experience, rarely addressed in a rigorous way. Given their importance among employers, educators should consider adding courses on these subjects.

\subsection{RQ2: Demonstration Through Open Source}

Although hiring managers had different thresholds and specifics, there were many commonalities in how they wanted students to demonstrate key traits through open source contributions. Employers wanted 6--10 issues solved at a variety of difficulty levels or in different applications or parts of an application (\textit{problem solving}), for students to solve problems that were not directly asked for (\textit{initiative}), and for students to help with documentation and participate in code reviews (\textit{technical communication}).

There was less consistency in how employers wanted students to demonstrate \textit{domain knowledge}, \textit{curiosity}, \textit{product sense}, or \textit{persistence}. However, based on our conversations we suspect that many hiring managers would still be enthusiastic about candidates who demonstrate these traits in other ways.

The desired \textit{domain knowledge} was strongly tied to the specific team for which the individual was hiring. CS degrees that offer tracks/concentration/focuses may have an advantage here.

\subsection{H1: Impact on Motivation of Students' Knowledge of Employer Expectations}

Students who viewed the hiring manager agreements were more motivated by the value of the open source contributions, particularly the Attainment Value. Students who viewed the hiring manager agreements also thought the costs of contributing were lower. (Cronbach's $\alpha$ was low for Utility Value, suggesting that it may be unreliable.) This finding is perhaps the most actionable for educators. It suggests that explicit discussion of hiring criteria with employers could improve student engagement with educational activities.